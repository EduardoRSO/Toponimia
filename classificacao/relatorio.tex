\documentclass{article}
\usepackage[utf8]{inputenc}
\usepackage[T1]{fontenc}
\usepackage{lmodern}
\usepackage{hyperref}

\title{Relatório final – MAC0215 – atividade curricular em pesquisa}
\author{Eduardo Ribeiro Silva de Oliveira}
\date{05 de Dezembro de 2024}

\begin{document}

\maketitle

\section*{Dados do aluno}
Nome: Eduardo Ribeiro Silva de Oliveira\\
Endereço de email: eduardo.rso784@usp.br\\
NUSP: 11796920

\section*{Título do projeto}
Classificação automatizada de topônimos: abordagens semânticas e morfológicas

\section*{Orientador}
Nome: Patricia de Jesus Carvalhinhos \\
Email: patricia.carv@usp.br\\
Endereço profissional: Faculdade de Filosofia, Letras e Ciências Humanas da Universidade de São Paulo (FFLCH-USP)

\section*{Resumo do projeto}
Este projeto tem como objetivo desenvolver um sistema de classificação automatizada de topônimos utilizando abordagens semânticas e morfológicas. A classificação será baseada na semelhança entre palavras, com foco na análise da presença de sufixos e no uso de sinônimos. A introdução do projeto abordará a importância de métodos eficazes para a categorização de topônimos em estudos linguísticos e geográficos. Os objetivos principais incluem a implementação de algoritmos que avaliem a semelhança lexical e a criação de regras específicas para a detecção e categorização de sufixos e sinônimos, visando melhorar a precisão na classificação dos topônimos.

\section*{Metodologia}
Não houve modelo de linguagem natural, apenas utilizamos a seguinte abordagem que suporta uma parametrização:

\begin{itemize}
    \item Verificar se a palavra possui um sufixo toponímico;
    \item Verificar se existem palavras similares no texto;
    \item Verificar se existe similaridade com os sinônimos;
    \item Verificar se existe similaridade com os topônimos do IBGE;
    \item Verificar se algum dos similares possui sufixo toponímico;
    \item Garantir que a pontuação esteja entre 0 e 1.
\end{itemize}

Assim, retornamos uma probabilidade da palavra ser um topônimo ou não.

A dificuldade para construir o modelo de linguagem natural é a base de treinamento: seriam necessários textos e os topônimos contidos no texto. No entanto, não encontrei essa base de dados pronta na internet, e construir uma excederia o escopo de 100 horas do projeto.

Para fazer o cálculo da confiança dos topônimos, utilizei uma base de dados com textos extraídos da Alesp, Prefeitura e a Câmara de Deputados. Os detalhes da construção desse dataframe constam no relatório de outro projeto desenvolvido neste semestre.

\section*{Código desenvolvido}
Optei por uma modularização seguindo a estrutura abaixo:
\begin{itemize}
    \item \texttt{classification\_workflow\_logic}: Coordena os demais módulos;
    \item \texttt{suffix\_logic}: Verifica se uma palavra possui sufixos toponímicos;
    \item \texttt{synonym\_logic}: Recupera os sinônimos de uma dada palavra;
    \item \texttt{lexical\_similarity\_logic}: Verifica a similaridade entre palavras;
    \item \texttt{ibge\_toponyms\_logic}: Base de dados com topônimos do IBGE (cidades e municípios).
\end{itemize}

\section*{O projeto}
O projeto foi realizado em várias etapas, conforme descrito abaixo:
\begin{enumerate}
\item \textbf{Revisão bibliográfica}: Pesquisa de estudos e metodologias existentes na classificação de topônimos, com foco em abordagens semânticas e morfológicas. Foram utilizados artigos acadêmicos e livros especializados para estabelecer uma base teórica sólida.
\item \textbf{Coleta e preparação de dados}: Reunião de um conjunto de dados de topônimos de fontes diversas, como documentos governamentais e redes sociais, seguido pela limpeza e preparação dos dados para garantir a qualidade do input para o modelo.
\item \textbf{Desenvolvimento do modelo}: Implementação de algoritmos de classificação baseados em semelhança lexical, identificação de sufixos e sinônimos. Utilizamos bibliotecas como spaCy, Word2Vec e WordNet para a construção do modelo.
\item \textbf{Treinamento do modelo}: Aplicação dos algoritmos aos dados coletados e treinamento do modelo para avaliar a precisão e eficácia da classificação, utilizando técnicas de embeddings semânticos e regras específicas de categorização.
\item \textbf{Documentação e apresentação}: Documentação detalhada de todas as etapas do projeto, além da preparação para a apresentação final dos resultados.
\end{enumerate}

\section*{Relatório final}
O projeto de classificação automatizada de topônimos foi extremamente enriquecedor, proporcionando tanto o desenvolvimento de habilidades técnicas em processamento de linguagem natural quanto o entendimento mais profundo sobre a categorização de topônimos. A aplicação de abordagens semânticas e morfológicas possibilitou a criação de um sistema eficaz de classificação, que se mostrou relevante para estudos linguísticos e geográficos.

Durante a execução do projeto, realizamos uma revisão bibliográfica para entender as metodologias existentes e definir as melhores abordagens a serem aplicadas. Em seguida, foi realizada a coleta de dados de diferentes fontes, seguida pela preparação dos mesmos, o que envolveu técnicas de limpeza e normalização. Na fase de desenvolvimento, utilizamos ferramentas como spaCy, Word2Vec e WordNet para implementar o modelo de classificação.

O treinamento do modelo foi uma etapa crucial, onde aplicamos os algoritmos desenvolvidos para categorizar os topônimos com base na semelhança lexical e na presença de sufixos e sinônimos. A documentação foi realizada utilizando a metodologia de auto-documentação proposta por Valdemar W. Setzer, garantindo que cada etapa do desenvolvimento fosse bem descrita e pudesse ser facilmente consultada.

As atividades resultaram em um total de 150 horas dedicadas ao projeto, contemplando todas as fases de produção. A experiência proporcionou um aprendizado significativo na área de processamento de linguagem natural e análise linguística, e os resultados obtidos incluem um sistema de classificação automatizado que pode ser utilizado por pesquisadores interessados na análise de topônimos, contribuindo para sua formação acadêmica e desenvolvimento profissional.

O código desenvolvido e os detalhes adicionais do projeto estão disponíveis no repositório público do GitHub: \url{https://github.com/EduardoRSO/Toponimia}.

\section*{Resultados obtidos}
A classificação utilizando uma abordagem semântica e morfológica revelou-se limitada para a definição precisa de um topônimo em um texto. A principal dificuldade encontrada foi a subjetividade inerente à toponímia, que varia conforme o contexto linguístico, histórico e cultural de cada palavra ou expressão.

Embora o sistema desenvolvido tenha conseguido identificar padrões morfológicos, como sufixos toponímicos, e relacioná-los semanticamente a outras palavras no texto, os resultados não foram suficientemente robustos para categorizar de forma confiável uma palavra como topônimo ou não. Essa limitação está diretamente associada à complexidade e à subjetividade do campo de estudo, onde fatores contextuais e interpretações subjetivas desempenham um papel crucial.

Apesar disso, o projeto proporcionou insights importantes sobre o uso de técnicas de processamento de linguagem natural aplicadas à análise de topônimos, destacando a necessidade de bases de dados mais abrangentes e específicas, bem como a importância de incorporar conhecimentos linguísticos mais profundos para melhorar a precisão e a eficácia do sistema.


\end{document}
