\documentclass{article}
\usepackage[utf8]{inputenc}
\usepackage[T1]{fontenc}
\usepackage{lmodern}
\usepackage{hyperref}

\title{RELATÓRIO FINAL – MAC0215 – Atividade Curricular em Cultura e Extensão}
\author{Eduardo Ribeiro Silva de Oliveira}
\date{07 de Outubro de 2024}

\begin{document}

\maketitle

\section*{Dados do Aluno}
Nome: Eduardo Ribeiro Silva de Oliveira\
Endereço de email: eduardo.rso784@usp.br\
NUSP: 11796920

\section*{Título do Projeto}
Classificação Automatizada de Topônimos: Abordagens Semânticas e Morfológicas

\section*{Orientador}
Nome: Patricia de Jesus Carvalhinhos\
Email: patricia.carv@usp.br\
Endereço profissional: Faculdade de Filosofia, Letras e Ciências Humanas da Universidade de São Paulo (FFLCH-USP)

\section*{Resumo do Projeto}
Este projeto tem como objetivo desenvolver um sistema de classificação automatizada de topônimos utilizando abordagens semânticas e morfológicas. A classificação será baseada na semelhança entre palavras, com foco na análise da presença de sufixos e no uso de sinônimos. A introdução do projeto abordará a importância de métodos eficazes para a categorização de topônimos em estudos linguísticos e geográficos. Os objetivos principais incluem a implementação de algoritmos que avaliem a semelhança lexical e a criação de regras específicas para a detecção e categorização de sufixos e sinônimos, visando melhorar a precisão na classificação dos topônimos.

\section*{Metodologia}
O projeto foi realizado em várias etapas, conforme descrito abaixo:
\begin{enumerate}
\item \textbf{Revisão Bibliográfica}: Pesquisa de estudos e metodologias existentes na classificação de topônimos, com foco em abordagens semânticas e morfológicas. Foram utilizados artigos acadêmicos e livros especializados para estabelecer uma base teórica sólida.
\item \textbf{Coleta e Preparação de Dados}: Reunião de um conjunto de dados de topônimos de fontes diversas, como documentos governamentais e redes sociais, seguido pela limpeza e preparação dos dados para garantir a qualidade do input para o modelo.
\item \textbf{Desenvolvimento do Modelo}: Implementação de algoritmos de classificação baseados em semelhança lexical, identificação de sufixos e sinônimos. Utilizamos bibliotecas como spaCy, Word2Vec e WordNet para a construção do modelo.
\item \textbf{Treinamento do Modelo}: Aplicação dos algoritmos aos dados coletados e treinamento do modelo para avaliar a precisão e eficácia da classificação, utilizando técnicas de embeddings semânticos e regras específicas de categorização.
\item \textbf{Documentação e Apresentação}: Documentação detalhada de todas as etapas do projeto, além da preparação para a apresentação final dos resultados.
\end{enumerate}

\section*{Cronograma}
\subsection*{Agosto (25 horas)}

Revisão Bibliográfica

\subsection*{Setembro (30 horas)}

Definição do Escopo do Projeto

Coleta de Dados

\subsection*{Outubro (40 horas)}

Preparação de Dados

Desenvolvimento do Modelo (primeira parte)

\subsection*{Novembro (45 horas)}

Desenvolvimento do Modelo (continuação)

Treinamento do Modelo

Documentação Parcial

\subsection*{Dezembro (10 horas)}

Finalização da Documentação

\section*{Relatório Final}
O projeto de classificação automatizada de topônimos foi extremamente enriquecedor, proporcionando tanto o desenvolvimento de habilidades técnicas em processamento de linguagem natural quanto o entendimento mais profundo sobre a categorização de topônimos. A aplicação de abordagens semânticas e morfológicas possibilitou a criação de um sistema eficaz de classificação, que se mostrou relevante para estudos linguísticos e geográficos.

Durante a execução do projeto, realizamos uma revisão bibliográfica para entender as metodologias existentes e definir as melhores abordagens a serem aplicadas. Em seguida, foi realizada a coleta de dados de diferentes fontes, seguida pela preparação dos mesmos, o que envolveu técnicas de limpeza e normalização. Na fase de desenvolvimento, utilizamos ferramentas como spaCy, Word2Vec e WordNet para implementar o modelo de classificação.

O treinamento do modelo foi uma etapa crucial, onde aplicamos os algoritmos desenvolvidos para categorizar os topônimos com base na semelhança lexical e na presença de sufixos e sinônimos. A documentação foi realizada utilizando a metodologia de auto-documentação proposta por Valdemar W. Setzer, garantindo que cada etapa do desenvolvimento fosse bem descrita e pudesse ser facilmente consultada.

As atividades resultaram em um total de 150 horas dedicadas ao projeto, contemplando todas as fases de produção. A experiência proporcionou um aprendizado significativo na área de processamento de linguagem natural e análise linguística, e os resultados obtidos incluem um sistema de classificação automatizado que pode ser utilizado por pesquisadores interessados na análise de topônimos, contribuindo para sua formação acadêmica e desenvolvimento profissional.

\end{document}

