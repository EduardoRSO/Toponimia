\documentclass{beamer}
\usepackage[utf8]{inputenc}
\usepackage[T1]{fontenc}
\usepackage{lmodern}

\title{Aula 5: Desenvolvendo Scrappers}
\author{Eduardo Ribeiro Silva de Oliveira}
\date{07 de Outubro de 2024}

\begin{document}

\frame{\titlepage}

% Slide 1: Estratégias de Coleta de Dados
\begin{frame}
  \frametitle{Estratégias de Coleta de Dados}
  \begin{itemize}
    \item Como identificar e selecionar fontes de dados.
    \item Importância de escolher fontes confiáveis e relevantes para o objetivo do scraping.
  \end{itemize}
\end{frame}

% Slide 2: Identificação de Sites Relevantes
\begin{frame}
  \frametitle{Identificação de Sites Relevantes}
  \begin{itemize}
    \item Identificar os sites que possuem informações relevantes para o objetivo do scraping.
    \item Considerar critérios como qualidade da informação, acessibilidade e atualização dos dados.
  \end{itemize}
\end{frame}

% Slide 3: Critérios para Seleção de Páginas
\begin{frame}
  \frametitle{Critérios para Seleção de Páginas}
  \begin{itemize}
    \item Estabelecer critérios para selecionar páginas ou seções que forneçam dados úteis.
    \item Avaliar a estrutura do site e a localização dos dados de interesse.
  \end{itemize}
\end{frame}

% Slide 4: Cuidados Legais e Éticos
\begin{frame}
  \frametitle{Cuidados Legais e Éticos}
  \begin{itemize}
    \item Respeitar a privacidade dos usuários e os termos de uso das plataformas.
    \item Entender as implicações legais de coletar dados de diferentes sites.
  \end{itemize}
\end{frame}

% Slide 5: Implementação de Scrappers com Selenium
\begin{frame}
  \frametitle{Implementação de Scrappers com Selenium}
  \begin{itemize}
    \item Aplicação das noções de HTML, JSON e programação para acessar sites e extrair dados.
    \item Selenium permite simular um navegador para interagir com elementos da página.
  \end{itemize}
\end{frame}

% Slide 6: Interação com Elementos Dinâmicos
\begin{frame}
  \frametitle{Interação com Elementos Dinâmicos}
  \begin{itemize}
    \item Navegação entre páginas, interações com botões, campos de pesquisa e outros elementos dinâmicos.
    \item Importância de compreender JavaScript para interagir com elementos carregados dinamicamente.
  \end{itemize}
\end{frame}

% Slide 7: Extração e Armazenamento dos Dados
\begin{frame}
  \frametitle{Extração e Armazenamento dos Dados}
  \begin{itemize}
    \item Extração e armazenamento dos dados coletados em JSON para uma estrutura organizada e acessível.
    \item JSON é amplamente utilizado por sua simplicidade e compatibilidade com várias linguagens.
  \end{itemize}
\end{frame}

% Slide 8: Testes e Validação dos Scrappers
\begin{frame}
  \frametitle{Testes e Validação dos Scrappers}
  \begin{itemize}
    \item Criação de rotinas de teste para verificar se os scrappers estão extraindo os dados esperados.
    \item Testes automatizados são importantes para garantir a robustez dos scrappers.
  \end{itemize}
\end{frame}

% Slide 9: Manutenção dos Scrappers
\begin{frame}
  \frametitle{Manutenção dos Scrappers}
  \begin{itemize}
    \item Como lidar com mudanças na estrutura dos sites que possam quebrar o scrapper.
    \item Atualização constante dos scrappers para acompanhar mudanças nos sites de interesse.
  \end{itemize}
\end{frame}

% Slide 10: Monitoramento e Logs
\begin{frame}
  \frametitle{Monitoramento e Logs}
  \begin{itemize}
    \item Dicas para monitorar o desempenho e registrar logs de execução.
    \item Logs ajudam a identificar erros e pontos de falha, facilitando a manutenção.
  \end{itemize}
\end{frame}

\end{document}