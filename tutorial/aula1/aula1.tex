\documentclass{beamer}
\usepackage[utf8]{inputenc}
\usepackage[T1]{fontenc}
\usepackage{lmodern}
\usepackage{graphicx}
\usepackage{listings}

\title{Conceitos Basicos de Programacao}
\author{Eduardo Ribeiro Silva de Oliveira}
\date{07 de Outubro de 2024}

\begin{document}

\frame{\titlepage}

% Slide 1: Introducao a Programacao
\begin{frame}
  \frametitle{Introducao a Programacao}
  \begin{itemize}
    \item O que e programacao?
    \begin{itemize}
      \item Programacao e o processo de criar instrucoes que um computador pode seguir para realizar tarefas especificas. Essas instrucoes sao escritas em linguagens de programacao, como Python, Java, ou C++.
    \end{itemize}
    \item Como a programacao funciona?
    \begin{itemize}
      \item A programacao funciona atraves da escrita de algoritmos que descrevem passo a passo o que o computador deve fazer. O codigo e transformado em linguagem de maquina, que e entendida pelo processador do computador.
    \end{itemize}
    \item Importancia da programacao em diferentes contextos.
    \begin{itemize}
      \item A programacao e importante porque esta presente em praticamente todos os aspectos da tecnologia moderna. Ela permite a criacao de softwares, sites, aplicativos e automatizacoes que facilitam nossas vidas e aumentam a produtividade.
    \end{itemize}
  \end{itemize}
\end{frame}

% Slide 2: Conceitos Basicos
\begin{frame}
  \frametitle{Conceitos Basicos}
  \begin{itemize}
    \item Variaveis: Sao como caixas que armazenam informacoes.
      \begin{itemize}
        \item Uma variavel pode armazenar diferentes tipos de valores, como numeros, texto, ou listas. Por exemplo, em Python, podemos definir uma variavel como `idade = 25` para armazenar a idade de uma pessoa.
      \end{itemize}
    \item Loops: Estruturas que repetem acoes ate uma condicao ser atendida.
      \begin{itemize}
        \item Loops sao usados para automatizar tarefas repetitivas. Por exemplo, podemos usar um `for` loop para iterar por uma lista de itens, ou um `while` loop para continuar uma acao enquanto uma determinada condicao for verdadeira.
      \end{itemize}
    \item Condicoes: Verificacoes que determinam qual caminho o programa deve seguir.
      \begin{itemize}
        \item Condicoes sao usadas para tomar decisoes no codigo. Por exemplo, podemos usar um `if` para verificar se um valor e maior que outro e, com base nisso, realizar uma acao especifica.
      \end{itemize}
  \end{itemize}
\end{frame}

% Slide 3: Primeiros Passos com Python
\begin{frame}[fragile]
  \frametitle{Primeiros Passos com Python}
  \begin{itemize}
    \item Instalando o Python e configurando o ambiente.
      \begin{itemize}
        \item O Python pode ser instalado a partir do site oficial (python.org). Tambem e recomendavel usar um ambiente virtual para gerenciar pacotes.
      \end{itemize}
    \item Escrevendo o primeiro programa: "Hello, World!".
      \begin{itemize}
        \item O classico primeiro programa e simples e imprime uma mensagem na tela.
      \end{itemize}
  \end{itemize}
  \tiny
\begin{verbatim}
print("Hello, World!")
\end{verbatim}
\end{frame}

% Slide 4: Inicializando uma Variavel em Python
\begin{frame}[fragile]
  \frametitle{Inicializando uma Variavel em Python}
  \tiny
\begin{lstlisting}
nome = "Eduardo"
idade = 22
\end{lstlisting}
\end{frame}

% Slide 5: Usando For Loop e While Loop
\begin{frame}[fragile]
  \frametitle{Usando For Loop e While Loop}
  \tiny
\begin{lstlisting}
for i in range(5):
    print("Este e o numero", i)

contador = 0
while contador < 5:
    print("Contador e", contador)
    contador += 1
\end{lstlisting}
\end{frame}

% Slide 6: Usando If, Elif e Else
\begin{frame}[fragile]
  \frametitle{Usando If, Elif e Else}
  \tiny
\begin{lstlisting}
x = 10
if x < 5:
    print("x e menor que 5")
elif x == 10:
    print("x e igual a 10")
else:
    print("x e maior que 5 e diferente de 10")
\end{lstlisting}
\end{frame}

% Slide 7: Manipulacao de Listas em Python
\begin{frame}[fragile]
  \frametitle{Manipulacao de Listas em Python}
  \tiny
\begin{lstlisting}
frutas = ["maca", "banana", "cereja"]
frutas.append("laranja")
print(frutas)

# Removendo um item
frutas.remove("banana")
print(frutas)
\end{lstlisting}
\end{frame}

% Slide 8: Manipulacao de Dicionarios em Python
\begin{frame}[fragile]
  \frametitle{Manipulacao de Dicionarios em Python}
  \tiny
\begin{lstlisting}
aluno = {"nome": "Eduardo", "idade": 22, "curso": "Ciencia da Computacao"}
print(aluno["nome"])

# Adicionando um novo par chave-valor
aluno["nota"] = 9.5
print(aluno)
\end{lstlisting}
\end{frame}

% Slide 9: Manipulacao de Strings em Python
\begin{frame}[fragile]
  \frametitle{Manipulacao de Strings em Python}
  \tiny
\begin{lstlisting}
texto = "Ola, Mundo!"
print(texto.upper())  # Converte para maiusculas
print(texto.lower())  # Converte para minusculas
print(texto.replace("Mundo", "Eduardo"))  # Substitui parte da string
\end{lstlisting}
\end{frame}

% Slide 10: Exemplo Completo
\begin{frame}[fragile]
  \frametitle{Exemplo Completo}
  \tiny
\begin{lstlisting}
# Inicializando dicionario com listas
dados = {
    "Alice": [25, "Feminino", "Inception"],
    "Bob": [30, "Masculino", "Matrix"],
    "Carlos": [28, "Masculino", "Star Wars"]
}

# Usando loops e condicoes
for nome, informacoes in dados.items():
    if nome == "Bob":
        print(f"""
        Nome: {nome},
        Idade: {informacoes[0]},
        Sexo: {informacoes[1]},
        Filme Favorito: {informacoes[2]}
        """)
\end{lstlisting}
\end{frame}

\end{document}