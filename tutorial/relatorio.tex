\documentclass{article}
\usepackage[utf8]{inputenc}
\usepackage[T1]{fontenc}
\usepackage{lmodern}
\usepackage{hyperref}

\title{Relatório final – MAC0213 - atividade curricular em cultura e extensão}
\author{Eduardo Ribeiro Silva de Oliveira}
\date{05 de dezembro de 2024}

\begin{document}

\maketitle

\section*{Dados do aluno}
Nome: Eduardo Ribeiro Silva de Oliveira\\
Endereço de email: eduardo.rso784@usp.br\\
NUSP: 11796920

\section*{Título do projeto}
Desenvolvimento de tutoriais acessíveis para ferramentas de automação e scraping

\section*{Orientador}
Nome: Patricia de Jesus Carvalhinhos\\
Email: patricia.carv@usp.br\\
Endereço profissional: Faculdade de Filosofia, Letras e Ciências Humanas da Universidade de São Paulo (FFLCH-USP)

\section*{Resumo do projeto}
Este projeto visa desenvolver uma série de tutoriais em vídeo voltados para capacitar pesquisadores no uso de ferramentas de automação e scraping, com foco em acessibilidade. A automação e o scraping são habilidades essenciais para otimizar a coleta e a análise de dados em pesquisas acadêmicas. No entanto, muitos pesquisadores enfrentam barreiras ao utilizar essas ferramentas devido à falta de tutoriais acessíveis e didáticos. Este projeto busca preencher essa lacuna, oferecendo materiais que sejam tanto tecnicamente precisos quanto fáceis de seguir, com ênfase em boas práticas de usabilidade e acessibilidade digital.

\section*{Metodologia}
O projeto foi desenvolvido em várias etapas, começando com uma pesquisa inicial sobre as principais ferramentas de automação e scraping utilizadas por pesquisadores. Em seguida, foram identificadas as principais barreiras de acessibilidade enfrentadas pelos usuários. Com base nessas informações, foram elaborados roteiros de tutoriais, que posteriormente foram gravados e editados. Finalmente, os tutoriais foram disponibilizados em uma plataforma online acessível, e foram coletados feedbacks para futuras melhorias.

\section*{Cronograma}
\subsection*{Agosto (20 horas)}
- Pesquisa inicial sobre ferramentas de automação e scraping.\\
- Revisão de materiais e preparação inicial do conteúdo.\\

\subsection*{Setembro (20 horas)}
- Identificação das barreiras de acessibilidade enfrentadas por pesquisadores.\\
- Coleta de informações e organização dos principais desafios.\\

\subsection*{Outubro (20 horas)}
- Elaboração dos roteiros para os tutoriais.\\
- Planejamento das gravações e definição dos tópicos a serem abordados.\\

\subsection*{Novembro (60 horas)}
- Gravação dos tutoriais (15 horas).\\
- Edição dos vídeos (15 horas).\\
- Ajustes finais e disponibilização dos tutoriais em uma plataforma acessível (20 horas).\\
- Coleta de feedback e relatório final do projeto (10 horas).\\

\section*{Resultados obtidos}
Os tutoriais desenvolvidos proporcionaram um material didático acessível e de alta qualidade, permitindo que pesquisadores superassem barreiras comuns ao uso de ferramentas de automação e scraping. Foram gravados e disponibilizados vídeos com explicações práticas, detalhadas e dinâmicas, cobrindo conceitos fundamentais e boas práticas.

Os resultados mostraram que a combinação de vídeos tutoriais, códigos exemplificativos e atenção à acessibilidade digital foi eficaz para capacitar pesquisadores. Entretanto, feedbacks indicaram a necessidade de materiais adicionais para atender a públicos com diferentes níveis de conhecimento técnico. Além disso, o projeto contribuiu para ampliar as discussões sobre acessibilidade em materiais educacionais no campo tecnológico.

Os tutoriais estão disponíveis no YouTube e podem ser acessados pelo seguinte link: \url{https://www.youtube.com/playlist?list=PL4zcPv8opqa8IKs9fX1HXq2nzCQj6C_Aj}. 

O código desenvolvido e os roteiros detalhados estão disponíveis no repositório público do GitHub: \url{https://github.com/EduardoRSO/Toponimia}.

\section*{Relatório final}
O projeto de desenvolvimento de tutoriais acessíveis para ferramentas de automação e scraping foi extremamente enriquecedor, proporcionando tanto o desenvolvimento de habilidades técnicas quanto pedagógicas. O objetivo principal de criar um material didático acessível para pesquisadores interessados em automação e scraping foi cumprido, com a produção de vídeos, roteiros e códigos exemplificativos.

Durante o desenvolvimento, foram criados roteiros detalhados para cada aula, que posteriormente foram convertidos em vídeos práticos para tornar o aprendizado mais dinâmico e interativo. Além disso, foram desenvolvidos códigos em Python, utilizando bibliotecas como BeautifulSoup e Selenium, para exemplificar os conceitos ensinados. Os tutoriais também enfatizaram boas práticas de usabilidade e acessibilidade digital, visando garantir que o conteúdo pudesse ser utilizado por todos, independentemente de possíveis limitações.

As atividades resultaram em um total de 120 horas dedicadas ao projeto, contemplando todas as fases de produção. A experiência proporcionou um aprendizado significativo na área de automação e coleta de dados, e também no que diz respeito ao ensino de tecnologias de forma prática e acessível. Os resultados obtidos incluem um material completo e de qualidade, que pode ser utilizado por pesquisadores para aprender scraping e automação de forma estruturada e prática, contribuindo para sua formação acadêmica e desenvolvimento profissional.

\end{document}
