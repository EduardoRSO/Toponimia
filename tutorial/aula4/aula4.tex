\documentclass{beamer}
\usepackage[utf8]{inputenc}
\usepackage[T1]{fontenc}
\usepackage{lmodern}

\title{Aula 4: Estruturando Dados com JSON}
\author{Eduardo Ribeiro Silva de Oliveira}
\date{07 de Outubro de 2024}

\begin{document}

\frame{\titlepage}

% Slide 1: O Que é JSON
\begin{frame}
  \frametitle{O Que é JSON}
  \begin{itemize}
    \item Introdução ao JSON e sua utilidade na organização de dados.
    \item JSON (JavaScript Object Notation) é um formato leve para troca de dados.
    \item É amplamente utilizado por ser de fácil leitura tanto para humanos quanto para máquinas.
  \end{itemize}
\end{frame}

% Slide 2: Estrutura de Dados
\begin{frame}
  \frametitle{Estrutura de Dados em JSON}
  \begin{itemize}
    \item JSON é composto por estruturas de chave-valor.
    \item Pode representar dicionários e listas:
    \begin{itemize}
      \item **Dicionários**: Estruturas de chave-valor, similares aos dicionários em Python.
      \item **Listas**: Conjuntos ordenados de valores, podendo conter outros objetos JSON.
    \end{itemize}
  \end{itemize}
\end{frame}

% Slide 3: Prática com JSON
\begin{frame}[fragile]
  \frametitle{Prática com JSON}
  \begin{itemize}
    \item Como criar e manipular dados em JSON.
    \item Exemplo de um objeto JSON:
    \begin{verbatim}
    {
      "nome": "Eduardo",
      "idade": 22,
      "habilidades": ["Python", 
      "Scraping", 
      "Data Analysis"]
    }
    \end{verbatim}
  \end{itemize}
\end{frame}

% Slide 4: Uso de JSON no Scraping
\begin{frame}
  \frametitle{Uso de JSON no Scraping}
  \begin{itemize}
    \item JSON é muito útil para armazenar dados coletados durante o scraping.
    \item Estruturar os dados em JSON facilita o armazenamento e posterior análise dos dados extraídos.
    \item A padronização do formato torna o processamento mais eficiente.
  \end{itemize}
\end{frame}

% Slide 5: Exemplos de Armazenamento com JSON
\begin{frame}[fragile]
  \frametitle{Exemplos de Armazenamento com JSON}
  \begin{itemize}
    \item Salvando dados coletados em um arquivo JSON:
    \begin{verbatim}
    import json

    dados = {
      "nome": "Eduardo",
      "idade": 22
    }

    with open("dados.json", "w") as arquivo:
      json.dump(dados, arquivo)
    \end{verbatim}
  \end{itemize}
\end{frame}

% Slide 6: Trabalhando com APIs e JSON
\begin{frame}
  \frametitle{Trabalhando com APIs e JSON}
  \begin{itemize}
    \item Muitas APIs retornam dados no formato JSON.
    \item O JSON facilita a extração e manipulação dos dados coletados através de uma API.
    \item Exemplo: Utilizando a biblioteca `requests` para acessar uma API e obter dados em JSON.
  \end{itemize}
\end{frame}

% Slide 7: Exemplo de Uso de API com JSON
\begin{frame}[fragile]
  \frametitle{Exemplo de Uso de API com JSON}
  \begin{verbatim}
  import requests

  resposta = requests.get("https://api.exemplo.com/dados")
  dados = resposta.json()
  print(dados["nome"])
  \end{verbatim}
\end{frame}

% Slide 8: Conversão para JSON
\begin{frame}
  \frametitle{Conversão para JSON}
  \begin{itemize}
    \item Como converter dados coletados em diferentes formatos para JSON.
    \item Garantir a padronização dos dados extraídos facilita seu uso em aplicações posteriores.
    \item Ferramentas e bibliotecas, como `json` em Python, ajudam na conversão de dados.
  \end{itemize}
\end{frame}

% Slide 9: Benefícios de Usar JSON
\begin{frame}
  \frametitle{Benefícios de Usar JSON}
  \begin{itemize}
    \item Formato leve e fácil de ler e escrever.
    \item Amplamente suportado por várias linguagens de programação.
    \item Facilita a integração entre diferentes sistemas e a análise de dados extraídos.
  \end{itemize}
\end{frame}

% Slide 10: Prática Final com JSON
\begin{frame}[fragile]
  \frametitle{Prática Final com JSON}
  \begin{itemize}
    \item Criação de um arquivo JSON a partir de dados coletados.
    \item Manipulação e leitura dos dados armazenados em JSON.
    \item Exemplo:
    \begin{verbatim}
    with open("dados.json", "r") as arquivo:
      conteudo = json.load(arquivo)
      print(conteudo["idade"])
    \end{verbatim}
  \end{itemize}
\end{frame}

\end{document}