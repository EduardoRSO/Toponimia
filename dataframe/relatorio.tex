\documentclass{article}
\usepackage[utf8]{inputenc}
\usepackage[T1]{fontenc}
\usepackage{lmodern}
\usepackage{hyperref}

\title{RELATÓRIO FINAL – MAC0214 – Atividade Curricular em Cultura e Extensão}
\author{Eduardo Ribeiro Silva de Oliveira}
\date{07 de Outubro de 2024}

\begin{document}

\maketitle

\section*{Dados do Aluno}
Nome: Eduardo Ribeiro Silva de Oliveira\\
Endereço de email: eduardo.rso784@usp.br\\
NUSP: 11796920

\section*{Título do Projeto}
Construção de um Dataframe de Topônimos Coletados da Internet

\section*{Orientador}
Nome: Patricia de Jesus Carvalhinhos\\
Email: patricia.carv@usp.br\\
Endereço profissional: Faculdade de Filosofia, Letras e Ciências Humanas da Universidade de São Paulo (FFLCH-USP)

\section*{Resumo do Projeto}
\textbf{Introdução:} O projeto visa a construção de um dataframe contendo topônimos, ou seja, nomes próprios de lugares, coletados de diferentes fontes na internet. A motivação central é a análise da variedade e frequência de topônimos em diferentes contextos, que pode ser utilizada para estudos linguísticos, históricos e socioculturais.\\
\textbf{Objetivos:}
\begin{itemize}
    \item Coletar dados de topônimos de diversas plataformas online, como sites de órgãos governamentais, redes sociais e plataformas de streaming.
    \item Organizar os dados em um dataframe estruturado, facilitando a análise e a visualização das informações coletadas.
    \item Analisar a distribuição e o uso de topônimos em diferentes contextos e regiões.
\end{itemize}

\section*{Metodologia}
O projeto foi desenvolvido em várias etapas:
\begin{enumerate}
    \item \textbf{Coleta de Dados:} A coleta de topônimos foi realizada em múltiplas etapas, utilizando ferramentas de scraping para extrair informações de fontes específicas, como o site da Câmara Municipal de São Paulo, Assembleia Legislativa, Senado Federal, Câmara dos Deputados, além de redes sociais como Facebook e X (antigo Twitter). Foram utilizados termos-chaves relevantes, como "denomina", "altera", "acrescenta", e "substitui" para capturar os dados necessários.
    \item \textbf{Construção do Dataframe:} Os dados coletados foram organizados em um dataframe utilizando Python e bibliotecas específicas para tratamento de dados, como pandas. O dataframe foi estruturado para permitir fácil consulta e análise posterior.
    \item \textbf{Análise dos Dados:} Com o dataframe consolidado, foi realizada uma análise estatística e visual dos topônimos, identificando padrões e possíveis tendências de uso.
\end{enumerate}

\section*{Cronograma}
\subsection*{Agosto (20 horas)}
- Pesquisa e definição de fontes de dados.
- Revisão de materiais e preparação inicial do conteúdo.

\subsection*{Setembro (50 horas)}
- Desenvolvimento de scripts de coleta de dados.
- Implementação de scrapers para as principais fontes identificadas.

\subsection*{Outubro (25 horas)}
- Coleta de dados e construção da primeira versão do dataframe.
- Ajustes nos scrapers para otimizar a coleta.

\subsection*{Novembro (55 horas)}
- Revisão e otimização do dataframe.
- Análise dos dados coletados.
- Conclusão da análise e redação do relatório final do projeto.

\section*{Relatório Final}
O projeto de construção de um dataframe de topônimos coletados da internet foi extremamente enriquecedor, proporcionando tanto o desenvolvimento de habilidades técnicas quanto a aplicação de conceitos de linguística e análise de dados. O objetivo principal de criar um material estruturado e acessível para estudos sobre topônimos foi cumprido, com a produção de um dataframe robusto que facilita análises diversas.

Durante o desenvolvimento, foram criados scrapers para diferentes fontes de dados, como órgãos governamentais e redes sociais, visando coletar topônimos de forma automatizada. Além disso, foram utilizadas técnicas de limpeza e organização dos dados para garantir a qualidade do dataframe final. A análise dos dados coletados permitiu identificar padrões de uso e distribuição dos topônimos, contribuindo para um melhor entendimento de suas ocorrências em diferentes contextos.

As atividades resultaram em um total de 150 horas dedicadas ao projeto, contemplando todas as fases de produção. A experiência proporcionou um aprendizado significativo na área de coleta de dados automatizada e no tratamento de dados para análise linguística. Os resultados obtidos incluem um material completo e de qualidade, que pode ser utilizado por pesquisadores interessados na análise de topônimos, contribuindo para sua formação acadêmica e desenvolvimento profissional.

\end{document}